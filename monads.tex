\documentclass{tufte-handout}

%\geometry{showframe}% for debugging purposes -- displays the margins

\usepackage{amsmath}

% Set up the images/graphics package
\usepackage{graphicx}
\setkeys{Gin}{width=\linewidth,totalheight=\textheight,keepaspectratio}
\graphicspath{{graphics/}}

% The following package makes prettier tables.  We're all about the bling!
\usepackage{booktabs}

% The units package provides nice, non-stacked fractions and better spacing
% for units.
\usepackage{units}

% The fancyvrb package lets us customize the formatting of verbatim
% environments.  We use a slightly smaller font.
\usepackage{fancyvrb}
\fvset{fontsize=\normalsize}

% Small sections of multiple columns
\usepackage{multicol}

% Provides paragraphs of dummy text
\usepackage{lipsum}

% These commands are used to pretty-print LaTeX commands
\newcommand{\doccmd}[1]{\texttt{\textbackslash#1}}% command name -- adds backslash automatically
\newcommand{\docopt}[1]{\ensuremath{\langle}\textrm{\textit{#1}}\ensuremath{\rangle}}% optional command argument
\newcommand{\docarg}[1]{\textrm{\textit{#1}}}% (required) command argument
\newenvironment{docspec}{\begin{quote}\noindent}{\end{quote}}% command specification environment
\newcommand{\docenv}[1]{\textsf{#1}}% environment name
\newcommand{\docpkg}[1]{\texttt{#1}}% package name
\newcommand{\doccls}[1]{\texttt{#1}}% document class name
\newcommand{\docclsopt}[1]{\texttt{#1}}% document class option name

\title{Monads Cheat Sheet}
\author[Adam Rosien]{Adam Rosien (\href{mailto:adam@rosien.net}{adam@rosien.net})}
\date{27 February 2013}  % if the \date{} command is left out, the current date will be used

\begin{document}

\maketitle% this prints the handout title, author, and date

%\printclassoptions

\begin{table}[ht]
  \centering
  \fontfamily{ppl}\selectfont
  \begin{tabular}{rllll}
    \toprule
    Monad        & Effect\footnotemark[1]  & Sequences as...                & \texttt{M[A]} \\ 
    \midrule
    Identity     & nothing                 & continue                       & \texttt{Id[A]} \\
    List         & arbitrary \# of results & halt if empty;                 & \texttt{List[A]} \\
                 &                         & continue with each value       & \\
    Option       & anonymous exception     & halt if \texttt{None};         & \texttt{Option[A]} \\
                 &                         & continue with value of \texttt{Some} & \\
    Either       & exception               & halt if \texttt{Left};         & \texttt{Either[L, R]} \\
                 &                         & continue with value of \texttt{Right} & \\
    Reader       & read-only state         & function composition           & \texttt{E => A} \\
    Writer       & append-only state       & append log value W             & \texttt{Writer[W, A]} \\
    State        & read+write state        & pass along the (updated) state & \texttt{State[S, A]} \\
    Continuation & continuation-passing    &                                & \texttt{Responder[A]} \\
    IO           & side-effect\footnotemark[2] & perform side-effect        & \texttt{IO[A]} \\
    \bottomrule
  \end{tabular}
  \label{tab:normaltab}
  %\zsavepos{pos:normaltab}
\end{table}

\footnotetext[1]{An \textit{effect} is \textbf{not} a \textit{side-effect}: the former is referentially-transparent while the latter is not.}
\footnotetext[2]{Wait, how can a side-effect become just an effect?}

\section{Monads by example}\label{sec:example}

Let's build up an idea of what monads do starting with an example:

\begin{quote} \ttfamily
for \{ \\
\ \ a <- Some(1) \\
\ \ b <- Some(10) \\
\} yield a + b

==> Some(11)
\end{quote}

Looks normal, right? Let's extract the \texttt{Some} constructors to a function named \texttt{pure}:

\begin{quote} \ttfamily
def pure[A](a: A): Option[A] = Some(a)

for \{ \\
\ \ a <- pure(1) \\
\ \ b <- pure(10) \\
\} yield a + b

==> Some(11)
\end{quote}

Now let's abstract the use of \texttt{Option} to any monad \texttt{M}:

\begin{quote} \ttfamily
def pure[A](a: A): M[A] = ... // TBD

for \{ \\
\ \ a <- pure(1) \\
\ \ b <- pure(10) \\
\} yield a + b

==> \textit{something to do with 11}
\end{quote}

Intuitively, the result of the expression should have something to do with 11.
For example, if you substitute back the type \texttt{Option} for \texttt{M}, \texttt{pure} should return a \texttt{Some},
and the result will be \texttt{Some(11)}.

\bibliography{sample-handout}
\bibliographystyle{plainnat}

\fancyfoot[C]{\copyright 2013 Adam S. Rosien (\href{mailto:adam@rosien.net}{\texttt{adam@rosien.net}}) \\ Source at \href{https://github.com/arosien/...}{\texttt{https://github.com/arosien/xxx}}}

\end{document}
